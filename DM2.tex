\documentclass[a4paper,10pt]{article}

\usepackage[utf8]{inputenc}
\usepackage[english]{babel}
\usepackage[T1]{fontenc}
\usepackage{mathpazo} %http://www.ctan.org/tex-archive/fonts/mathpazo
\usepackage{stmaryrd} %http://www.ctan.org/pkg/stmaryrd
\usepackage{amsmath} %http://www.ctan.org/pkg/amsmath
\usepackage{amssymb}
\usepackage{mathrsfs}

\usepackage{amsthm} %http://www.ctan.org/pkg/amsthm
\usepackage{proof}

\usepackage[colorlinks=true]{hyperref} %http://www.ctan.org/tex-archive/macros/latex/contrib/hyperref/
\hypersetup{urlcolor=black,linkcolor=black}

\usepackage{footmisc} %http://www.ctan.org/tex-archive/macros/latex/contrib/footmisc

\usepackage{enumerate}
\usepackage{ulem} %http://www.ctan.org/tex-archive/macros/latex/contrib/ulem
\normalem
\usepackage{cancel} %http://www.ctan.org/tex-archive/macros/latex/contrib/cancel

\usepackage{fullpage} %http://www.ctan.org/tex-archive/macros/latex/contrib/preprint/
\setlength{\parindent}{0pt}
\setlength{\parskip}{\medskipamount}

\usepackage{pgffor}
\usepackage{tikz}
\usetikzlibrary{shapes.arrows, chains, positioning, automata, graphs}
\usepackage{graphviz}

\usepackage[ruled,vlined,english]{algorithm2e}
\providecommand{\SetAlgoLined}{\SetLine}
\providecommand{\DontPrintSemicolon}{\dontprintsemicolon}

\usepackage{forest}
\usepackage{comment} %http://www.ctan.org/tex-archive/macros/latex/contrib/comment
\usepackage{multirow} %http://www.ctan.org/tex-archive/macros/latex/contrib/multirow
\usepackage{slashbox} %http://www.ctan.org/tex-archive/macros/latex/contrib/slashbox

\usepackage{listings} %http://www.ctan.org/tex-archive/macros/latex/contrib/listings/
\lstset{numbers=left,language=Caml}

\newcounter{ThComp}[section]
\newcounter{DefComp}

\newtheorem{definition}[ThComp]{Definition}
\newtheorem{theoreme}[ThComp]{Theorem}
\newtheorem{proposition}[ThComp]{Proposition}

\newtheorem*{fact}{Fact}
\newtheorem*{csq}{Consequence}
\newtheorem{thm}[ThComp]{Theorem}
\newtheorem{theorem}[ThComp]{Theorem}
\newtheorem{propo}[ThComp]{Proposition}
%\newtheorem{proposition}[ThComp]{Proposition}
\newtheorem{lemma}[ThComp]{Lemma}
\newtheorem*{corol}{Corollary}
\newtheorem{prop}[ThComp]{Property}
\newtheorem{property}[ThComp]{Property}
\theoremstyle{definition}
\newtheorem*{ex}{Example}
\newtheorem*{exs}{Examples}
\newtheorem{exo}{Exercise}
\newtheorem{defi}[DefComp]{Definition}
%\newtheorem{definition}[DefComp]{Definition}
\newtheorem{algo}{Algorithm}
\theoremstyle{remark}
\newtheorem*{Rq}{Remark}

\newcommand{\R}{\texttt{R}}

\newcommand{\RR}{\mathbb{R}}
\newcommand{\ZZ}{\mathbb{Z}}
\newcommand{\NN}{\mathbb{N}}
\newcommand{\PP}{\mathbb{P}}
\newcommand{\EE}{\mathbb{E}}
\newcommand{\IE}{\mathbb{E}}
\newcommand{\IR}{\mathbb{R}}
\newcommand{\IZ}{\mathbb{Z}}
\newcommand{\IN}{\mathbb{N}}
\newcommand{\IP}{\mathbb{P}}

\newcommand{\cN}{\mathcal{N}}
\newcommand{\cF}{\mathcal{F}}
\newcommand{\ck}{\mathcal{K}}
\newcommand{\cNU}{\mathcal{NU}}
\newcommand{\cL}{\mathcal{L}}
\newcommand{\N}{\mathcal{N}}
\newcommand{\F}{\mathcal{F}}
\renewcommand{\L}{\mathcal{L}}
\renewcommand{\H}{\mathcal{H}}

\newcommand{\ens}[1]{\left\{ #1 \right\}}
\newcommand{\set}[1]{\left\{ #1 \right\}}
\renewcommand{\leq}{\leqslant}
\renewcommand{\geq}{\geqslant}
\newcommand{\cplx}[1]{\mathcal O \left( #1 \right)}
\newcommand{\floor}[1]{\left \lfloor #1 \right \rfloor}
\newcommand{\ceil}[1]{\left\lceil #1 \right\rceil}
\newcommand{\donne}{\rightarrow}
\newcommand{\gives}{\rightarrow}
\newcommand{\dans}{\to}
\newcommand{\booleen}{\set{0,1}^*}
\newcommand{\eps}{\varepsilon}
\renewcommand{\implies}{~\Rightarrow~}
\newcommand{\tildarrow}{\rightsquigarrow}
\newcommand{\blank}{\texttt{\char32}}
\newcommand{\trans}[1]{\xrightarrow{#1}}
\newcommand{\rules}[1]{\xrightarrow{#1}}
\newcommand{\todo}[1]{\Large\textcolor{red}{#1}\normalsize}
\newcommand{\argmin}{\text{argmin}}
\newcommand{\bottom}{\bot}


%EvalPerf
\newcommand{\Var}{\text{Var}}
\newcommand{\prob}[1]{\PP\left( #1 \right)}


%SystDist
\newcommand{\Receive}{\texttt{Receive~}}
\newcommand{\Send}{\texttt{Send~}}


%Preuves
\newcommand{\betared}{\vartriangleright_\beta}
\newcommand{\parabetared}{\vartriangleright_{||\beta}}


%Cplx
\newcommand{\dtime}{\textsc{DTime}}
\newcommand{\dTime}{\textsc{DTime}}
\newcommand{\DTime}{\textsc{DTime}}

\newcommand{\ntime}{\textsc{NTime}}
\newcommand{\nTime}{\textsc{NTime}}
\newcommand{\NTime}{\textsc{NTime}}

\renewcommand{\P}{\textsc{P}}

\newcommand{\pTime}{\textsc{PTime}}
\newcommand{\PTime}{\textsc{PTime}}

\newcommand{\NP}{\textsc{NP}}

\newcommand{\npTime}{\textsc{NPTime}}
\newcommand{\NPTime}{\textsc{NPTime}}

\newcommand{\EXP}{\textsc{Exp}}
\newcommand{\expTime}{\textsc{Exp}}
\newcommand{\ExpTime}{\textsc{Exp}}
\newcommand{\EXPTime}{\textsc{Exp}}

\newcommand{\Space}{\textsc{Space}}

\newcommand{\dSpace}{\textsc{DSpace}}
\newcommand{\DSpace}{\textsc{DSpace}}


\newcommand{\nSpace}{\textsc{NSpace}}\newcommand{\NSpace}{\textsc{NSpace}}

\newcommand{\pSpace}{\textsc{PSpace}}
\newcommand{\PSpace}{\textsc{PSpace}}

\newcommand{\npSpace}{\textsc{NPSpace}}
\newcommand{\NpSpace}{\textsc{NPSpace}}
\newcommand{\NPSpace}{\textsc{NPSpace}}

\newcommand{\SpaceTM}{\textsc{SpaceTM}}

\newcommand{\nL}{\textsc{NL}}
\newcommand{\NL}{\textsc{NL}}

\newcommand{\LL}{\textsc{L}}

\newcommand{\coNP}{co\text{-}\textsc{NP}}

\newcommand{\conL}{co\text{-}\textsc{NL}}
\newcommand{\coNL}{co\text{-}\textsc{NL}}

\newcommand{\npc}{\text{\textit{NP-C}}}





\author{
    Marc \textsc{Chevalier}\\
    Thomas \textsc{Pellissier} \textsc{Tanon}}
\date{\today}
\title{\textsc{Friedman}'s translation}

\begin{document}

\maketitle

\section{\textsc{Friedman}'s Translation}

\begin{definition}
    Let $\R$ be a formula. The \textbf{parametrized negation} is 
    $$
        \ngr := A \Rightarrow \R
    $$
\end{definition}

We gather here some basic properties of the parametrized negation

\begin{proposition}
    In intuitionisctic logic,
    \begin{enumerate}[(i)]
        \item $B \Rightarrow \ngr A \vdash A \Rightarrow \ngr B$
        \item $A\vdash \ngr\ngr A$
        \item $A\Rightarrow B \vdash \ngr B \Rightarrow \ngr A$
        \item $A\Rightarrow B \vdash \ngr\ngr A \Rightarrow \ngr \ngr B$
        \item $\ngr \ngr \ngr A \vdash \ngr A$
    \end{enumerate}
\end{proposition}
\begin{proof}
        \begin{enumerate}[(i)]
        \item 
            $$
                \infer{
                    B \Rightarrow \neg_\R A \vdash A \Rightarrow \neg_\R B
                }{
                    \infer{
                        B \Rightarrow \neg_\R A, A \vdash \neg_\R B
                    }{
                        \infer{
                            B \Rightarrow \neg_\R A, A, B \vdash \R
                        }{
                            \infer{
                                B \Rightarrow \neg_\R A, A, B \vdash A
                            }{
                            }&
                            \infer{
                                B \Rightarrow \neg_\R A, A, B \vdash A\Rightarrow \R
                            }{
                                \infer{
                                    B \Rightarrow \neg_\R A, A, B \vdash B
                                }{
                                }&
                                \infer{
                                    B \Rightarrow \neg_\R A, A, B \vdash B \Rightarrow A \Rightarrow R
                                }{
                                }
                            }
                        }
                    }
                }
            $$
            
        \item 
            $$
                \infer{
                    A\vdash \neg_\R\neg_\R A
                }{
                    \infer{
                        A, \neg_\R A \vdash \R
                    }{
                        \infer{
                            A, \neg_\R A \vdash A\Rightarrow \R
                        }{
                        }&
                        \infer{
                            A, \neg_\R A \vdash A
                        }{
                        }
                    }
                }
            $$
        \item
            $$
                \infer{
                    A\Rightarrow B \vdash \neg_\R B \Rightarrow \neg_\R A
                }{
                    \infer{
                        A\Rightarrow B, \neg_\R B \vdash \neg_\R A
                    }{
                        \infer{
                            A\Rightarrow B, \neg_\R B,A \vdash \R
                        }{
                            \infer{
                                A\Rightarrow B, \neg_\R B,A \vdash B\Rightarrow \R
                            }{
                            }&
                            \infer{
                                A\Rightarrow B, \neg_\R B,A \vdash B
                            }{
                                \infer{
                                    A\Rightarrow B, \neg_\R B,A \vdash A\Rightarrow B
                                }{
                                }&
                                \infer{
                                    A\Rightarrow B, \neg_\R B,A \vdash A
                                }{
                                }
                            }
                        }
                    }
                }
            $$
        \item
            $$
                \infer{
                                        (\Pi_0) : A\Rightarrow B, \neg_\R\neg_\R A,\neg_\R B, A\vdash B
                                    }{
                                        \infer{
                                            A\Rightarrow B, \neg_\R\neg_\R A,\neg_\R B, A\vdash A\Rightarrow B
                                        }{
                                        }&
                                        \infer{
                                            A\Rightarrow B, \neg_\R\neg_\R A,\neg_\R B, A\vdash A
                                        }{
                                        }
                                    }
            $$

            \bigskip            
            
            $$
                \infer{
                                (\Pi_1) : A\Rightarrow B, \neg_\R\neg_\R A,\neg_\R B \vdash A\Rightarrow \R
                            }{
                                \infer{
                                    A\Rightarrow B, \neg_\R\neg_\R A,\neg_\R B, A\vdash \R
                                }{
                                    \infer{
                                        A\Rightarrow B, \neg_\R\neg_\R A,\neg_\R B, A\vdash B \Rightarrow \R
                                    }{
                                    }&
                                    \infer{
                                        A\Rightarrow B, \neg_\R\neg_\R A,\neg_\R B, A\vdash B
                                    }{
                                        \Pi_0
                                    }
                                }
                            }
            $$
            
            \bigskip
            
            $$
                \infer{
                    A\Rightarrow B \vdash \neg_\R\neg_\R A \Rightarrow \neg_\R \neg_\R B
                }{
                    \infer{
                        A\Rightarrow B, \neg_\R\neg_\R A \vdash \neg_\R \neg_\R B
                    }{
                        \infer{
                            A\Rightarrow B, \neg_\R\neg_\R A,\neg_\R B \vdash \R 
                        }{
                            \infer{
                                A\Rightarrow B, \neg_\R\neg_\R A,\neg_\R B \vdash A\Rightarrow \R \Rightarrow \R 
                            }{
                            }&
                            \infer{
                                A\Rightarrow B, \neg_\R\neg_\R A,\neg_\R B \vdash A\Rightarrow \R
                            }{
                                \Pi_1
                            }
                        }
                    }
                }
            $$
        \item
            $$
                            \infer{
                                (\Pi_2) : \neg_\R \neg_\R \neg_\R A, A, A\Rightarrow\R \vdash \R
                            }{
                                \infer{
                                    \neg_\R \neg_\R \neg_\R A, A, A\Rightarrow\R \vdash A\Rightarrow\R
                                }{
                                }&
                                \infer{
                                    \neg_\R \neg_\R \neg_\R A, A, A\Rightarrow\R \vdash A
                                }{
                                }
                            }
            $$        
        
            $$
                \infer{
                    \neg_\R \neg_\R \neg_\R A \vdash \neg_\R A
                }{
                    \infer{
                        \neg_\R \neg_\R \neg_\R A, A \vdash \R
                    }{
                        \infer{
                            \neg_\R \neg_\R \neg_\R A, A \vdash A\Rightarrow\R\Rightarrow\R\Rightarrow\R
                        }{
                        }&
                        \infer{
                            \neg_\R \neg_\R \neg_\R A, A \vdash A\Rightarrow\R\Rightarrow\R
                        }{
                            \infer{
                                \neg_\R \neg_\R \neg_\R A, A, A\Rightarrow\R \vdash \R
                            }{
                                \Pi_2
                            }
                        }
                    }
                }
            $$
    \end{enumerate}
\end{proof}

We now define the parametrized translation.
\begin{definition}
    Let $\R$ be a formula. The \textbf{parametrized negative translation} $A^\ngr$ is defined by induction on $A$ as follows:
    \begin{center}
    \begin{tabular}{cccc}
        $\bottom^\ngr := \R$ & $\top^\ngr := \top$ & \multicolumn{2}{c}{$(a\dot{=}b)^\ngr := \ngr\ngr(a\dot{=}b)$}\\
        \multicolumn{2}{c}{$(A\wedge B)^{\ngr}:=A^\ngr \wedge B^\ngr$} & \multicolumn{2}{c}{$(A \Rightarrow B)^\ngr:=A^\ngr \Rightarrow B^\ngr$}\\
        \multicolumn{4}{c}{$(A \vee B)^\ngr:=\ngr\ngr(A^\ngr \vee B^\ngr)$}\\
        \multicolumn{2}{c}{$\forall x A)^\ngr := \forall x A^\ngr$} & \multicolumn{2}{c}{$\exists x A^\ngr := \ngr\ngr(\exists x A^\ngr)$}
    \end{tabular}
    \end{center}
\end{definition}

Note that $(\neg A)^\ngr = \ngr A^\ngr$. We gather here the basic properties of the parametrized translation.

\begin{proposition}
    In intuitionistic logic,
    \begin{enumerate}[(i)]
        \item $\vdash (A\vee \neg A)^\ngr$
        \item $\R \vdash A^\ngr$
        \item $\ngr\ngr A^\ngr \vdash A^\ngr$
    \end{enumerate}
\end{proposition}
\begin{proof}
    \begin{enumerate}[(i)]
    \item We proof this property by induction.
    We will use intensively these lemmas:
    $$
    \infer{
        (\Psi_1) : \neg A \vdash \ngr A
    }{
        \infer{
            \neg A, A \vdash \R
        }{
            \infer{
                \neg A, A \vdash \bot
            }{
                \infer{
                    \neg A, A \vdash \neg A
                }{
                }&
                \infer{
                    \neg A, A \vdash A
                }{
                }
            }
        }
    }
    $$
    
    $$
    \infer{
        (\Psi_2) : A \vdash \ngr \ngr A
    }{
        \infer{
            A, \ngr A \vdash \R
        }{
            \infer{
                A, \ngr A \vdash A\Rightarrow \R
            }{
            }&
            \infer{
                A, \ngr A \vdash A
            }{
            }
        }
    }
    $$
    
    $$
    \infer{
        (\Psi_3) :\: \vdash \ngr \ngr A
    }{
        \infer{
             \ngr A \vdash \R
        }{
            \infer{
                 \ngr A \vdash A\Rightarrow \R
            }{
            }&
            \ngr A \vdash A
        }
    }
    $$
    
    We will often use this last one with the weakening
    
    \begin{itemize}
        \item $A = \top$.
            $$
            \infer{
                \vdash (\top \vee \neg \top)^\ngr
            }{
                \infer{
                    \vdash (\ngr \top) \vee (\ngr \bot)
                }{
                    \infer{
                        \vdash (\ngr \bot)
                    }{
                        \infer{
                            \bot \vdash \R
                        }{
                            \infer{
                                \bot \vdash \bot
                            }{
                            }
                        }
                    }
                }
            }
            $$
        \item $A = \bot$.
            $$
            \infer{
                \vdash (\bot \vee \neg \bot)^\ngr
            }{
                \infer{
                    \vdash (\bot \Rightarrow \R) \vee (\ngr \top)
                }{
                    \infer{
                        \vdash (\ngr \bot)
                    }{
                        \infer{
                            \bot \vdash \R
                        }{
                            \infer{\bot \vdash \bot}{}
                        }
                    }
                }
            }
            $$

        \item $A = (a \dot{=} b)$.

            $$
            \infer{
                (\Pi_3) :\:\vdash (a \dot{=} b) \vee (\ngr(a \dot{=} b))
            }{
                \infer{
                    \vdash (a \dot{=} b) \vee (\neg(a \dot{=} b))
                }{
                    \text{decidability of QF formulas}
                }
                &
                \infer{
                    (a \dot{=} b) \vdash (a \dot{=} b)
                }{
                }
                &
                \infer{
                    \neg(a \dot{=} b) \vdash \ngr(a \dot{=} b)
                }{
                    \Psi_1
                }
            }
            $$

            $$
            \infer{
                \vdash ((a \dot{=} b) \vee \neg(a \dot{=} b))^\ngr
            }{
                \infer{
                    \vdash \ngr\ngr(a \dot{=} b) \vee (\ngr\ngr\ngr(a \dot{=} b))
                }{
                    \infer{
                        (a \dot{=} b) \vdash \ngr\ngr(a \dot{=} b) \vee (\ngr\ngr\ngr(a \dot{=} b))
                    }{
                        \infer{
                            (a \dot{=} b) \vdash \ngr\ngr(a \dot{=} b)
                        }{
                            \Psi_2
                        }
                    }&
                    \infer{
                        \ngr(a \dot{=} b) \vdash \ngr\ngr(a \dot{=} b) \vee (\ngr\ngr\ngr(a \dot{=} b))
                    }{
                        \infer{
                            \ngr(a \dot{=} b) \vdash \ngr\ngr\ngr(a \dot{=} b)
                        }{
                            \Psi_2
                        }
                    }&
                    \Pi_3
                }
            }
            $$
        \item $A = B \vee C$
        
            $IH_B : \; \vdash (B \vee \neg B)^\ngr$
            
            $IH_C : \; \vdash (C \vee \neg C)^\ngr$

            $$
            \infer{
                (\Pi_6) : \ngr B^\ngr, \ngr C^\ngr, (B^\ngr \vee C^\ngr) \vdash \R
            }{
                    \infer{B^\ngr \vee C^\ngr \vdash B^\ngr \vee C^\ngr
                }{
                }&
                \infer{
                    \ngr B^\ngr, \ngr C^\ngr, B^\ngr \vdash \R
                }{
                    \infer{
                        \ngr B^\ngr, \ngr C^\ngr, B^\ngr \vdash B^\ngr
                    }{
                    }
                }&
                \infer{
                    \ngr B^\ngr, \ngr C^\ngr, C^\ngr \vdash \R
                }{
                    \infer{
                        \ngr B^\ngr, \ngr C^\ngr, C^\ngr \vdash C^\ngr
                    }{
                    }
                }
            }
            $$
            
            $$
                \infer{
                    (\Pi_7) : \ngr B^\ngr, \ngr C^\ngr \vdash \ngr\ngr(B^\ngr \vee C^\ngr) \vee \ngr\ngr\ngr (B^\ngr \vee C^\ngr)
                }{
                    \infer{
                        \ngr B^\ngr, \ngr C^\ngr \vdash \ngr\ngr\ngr (B^\ngr \vee C^\ngr)
                    }{
                        \infer{
                            \ngr B^\ngr, \ngr C^\ngr, \ngr\ngr (B^\ngr \vee C^\ngr) \vdash \R
                        }{
                            \infer{
                                \ngr B^\ngr, \ngr C^\ngr \vdash \ngr (B^\ngr \vee C^\ngr)
                            }{
                                \infer{
                                    \ngr B^\ngr, \ngr C^\ngr, (B^\ngr \vee C^\ngr) \vdash \R
                                }{
                                    \Pi_6
                                }
                            }
                        }
                    }
                }
            $$

            $$
            \infer{
                (\Pi_8) : \ngr B^\ngr \vdash \ngr\ngr(B^\ngr \vee C^\ngr) \vee \ngr\ngr\ngr (B^\ngr \vee C^\ngr)
            }{
                \infer{
                    \vdash (C \vee \ngr C)
                }{
                        \infer{\vdash (C \vee \neg C)^\ngr
                    }{
                    }
                }&
                \infer{
                    \ngr B^\ngr, C^\ngr \vdash \ngr\ngr(B^\ngr \vee C^\ngr) \vee \ngr\ngr\ngr (B^\ngr \vee C^\ngr
                }{
                    \infer{
                        \ngr B^\ngr, C^\ngr \vdash \ngr\ngr(B^\ngr \vee C^\ngr)
                    }{
                        \infer
                        {
                            \Psi_3
                        }{
                            \infer{
                                \ngr B^\ngr, C^\ngr, \ngr(B^\ngr \vee C^\ngr)\vdash (B^\ngr \vee C^\ngr)
                            }{
                                \infer{
                                    \ngr B^\ngr, C^\ngr, \ngr(B^\ngr \vee C^\ngr)\vdash C^\ngr)
                                }{
                                }
                            }
                        }
                    }
                }&
                \Pi_7
            }
            $$
            
            $$
                    \infer{
                        (\Pi_9) : \ngr B^\ngr \vdash \ngr\ngr(B^\ngr \vee C^\ngr) \vee \ngr\ngr\ngr (B^\ngr \vee C^\ngr)
                    }{
                        \Pi_8
                    }
            $$

            $$
            \infer{
                \vdash (B \vee C)^\ngr \vee (\ngr (B \vee C)^\ngr)
            }{
                \infer{
                    \vdash \ngr\ngr(B^\ngr \vee C^\ngr) \vee \ngr\ngr\ngr (B^\ngr \vee C^\ngr)
                }{
                    \infer{
                        \vdash B^\ngr \vee \neg B^\ngr
                    }{
                        \infer{
                            \vdash (B \vee \ngr B)^\ngr
                        }{
                        }
                    }&
                    \infer{
                        B^\ngr \vdash \ngr\ngr(B^\ngr \vee C^\ngr) \vee \ngr\ngr\ngr (B^\ngr \vee C^\ngr)
                    }{
                        \infer{
                            B^\ngr \vdash \ngr\ngr(B^\ngr \vee C^\ngr)
                        }{
                            \infer{
                                \Psi_3
                            }{
                                \infer{
                                    B^\ngr, \ngr(B^\ngr \vee C^\ngr) \vdash B^\ngr \vee C^\ngr
                                }{
                                    \infer{
                                        B^\ngr, \ngr(B^\ngr \vee C^\ngr) \vdash B^\ngr
                                    }{
                                    }
                                }
                            }
                        }
                    }&
                    \Pi_9
                }
            }
            $$

        \item $A = B \wedge C$
        
            $(IH_B) : \; \vdash (B \vee \neg B)^\ngr$
            
            $(IH_C) : \; \vdash (C \vee \neg C)^\ngr$


            $$
                \infer{
                    (\Pi_4) : B^\ngr, \ngr C^\ngr \vdash (B^\ngr \wedge C^\ngr) \vee \ngr(B^\ngr \wedge C^\ngr)
                }{
                    \infer{
                        B^\ngr, \neg C^\ngr \vdash \ngr(B^\ngr \wedge C^\ngr)
                    }{
                        \infer{
                            B^\ngr, \neg C^\ngr, B^\ngr \wedge C^\ngr \vdash \R
                        }{
                            \infer{
                                B^\ngr, B^\ngr \wedge C^\ngr \vdash C^\ngr
                            }{
                                \infer{
                                    B^\ngr, B^\ngr \wedge C^\ngr \vdash B^\ngr \wedge C^\ngr
                                }{
                                }
                            }
                        }
                    }
                }
            $$


            $$
            \infer{
                \Pi_5 : B^\ngr \vdash (B^\ngr \wedge C^\ngr) \vee \ngr(B^\ngr \wedge C^\ngr)
            }{
                \infer{
                    \vdash (C^\ngr \vee \ngr C^\ngr)
                }{
                    \infer{
                        \vdash (C \vee \neg C)^\ngr
                    }{
                        IH_C
                    }
                }&
                \infer{
                    B^\ngr, C^\ngr \vdash (B^\ngr \wedge C^\ngr) \vee \ngr(B^\ngr \wedge C^\ngr)
                }{
                    \infer{
                        B^\ngr, C^\ngr \vdash B^\ngr \wedge C^\ngr
                    }{
                        \infer{
                            B^\ngr, C^\ngr \vdash B^\ngr
                        }{
                        }&
                        \infer{
                            B^\ngr, C^\ngr \vdash C^\ngr
                        }{
                        }
                    }
                }&
                \Pi_4
            }
            $$

            $$
            \infer{
                \vdash ((B \wedge C) \vee \neg(B \wedge C))^\ngr
            }{\infer{\Psi_3}{
                    \infer{\vdash (B^\ngr \wedge C^\ngr) \vee \ngr(B^\ngr \wedge C^\ngr)
                }{
                        \infer{\vdash B^\ngr \vee \ngr B^\ngr
                    }{
                            \infer{\vdash (B \vee \neg B)^\ngr
                        }{
                            IH_B
                        }
                    }
                    &
                        \infer{B^\ngr \vdash (B^\ngr \wedge C^\ngr) \vee \ngr(B^\ngr \wedge C^\ngr)
                    }{
                        \Pi_5
                    }
                    &
                    \infer{
                        \neg B^\ngr \vdash (B^\ngr \wedge C^\ngr) \vee \ngr(B^\ngr \wedge C^\ngr)
                    }{
                        \infer{
                            \neg B^\ngr \vdash \ngr(B^\ngr \wedge C^\ngr)
                        }{
                            \infer{
                                \neg B^\ngr, B^\ngr \wedge C^\ngr \vdash \R
                            }{
                                \infer{
                                    B^\ngr \wedge C^\ngr \vdash B^\ngr
                                }{
                                    \infer{
                                        B^\ngr \wedge C^\ngr \vdash B^\ngr \wedge C^\ngr
                                    }{
                                    }
                                }
                            }
                        }
                    }
                }
            }}
            $$
        \item $A = B \Rightarrow C$
        
            $(IH_B) : \; \vdash (B \vee \neg B)^\ngr$
            
            $(IH_C) : \; \vdash (C \vee \neg C)^\ngr$
            
            $$
            $$
    \end{itemize}

    \item We proof this property by induction on $A$ and we denote by $IH$ the induction hypothesis.
    \begin{itemize}
        \item $A=\bottom$
            $$
                \infer{
                    \R \vdash \R
                }{
                }
            $$
        \item $A=\top$
            $$
                \infer{
                    \R \vdash \top
                }{
                }
            $$
        \item $A=(a \dot{=} b)$
            $$
                \infer{
                    \R \vdash (a \dot{=} b)^\ngr
                }{
                    \infer{
                        \R, \ngr(a \dot{=} b) \vdash \R
                    }{
                    }
                }
            $$
        \item $A=B\wedge C$
            $$
                \infer{
                    \R \vdash B^\ngr \wedge C^\ngr
                }{
                    \infer{
                        \R \vdash B^\ngr
                    }{
                        IH
                    }&
                    \infer{
                        \R \vdash C^\ngr
                    }{
                        IH
                    }
                }
            $$
        \item $A=B \Rightarrow C$
            $$
                \infer{
                    \R \vdash B^\ngr \Rightarrow C^\ngr
                }{
                    \infer{
                        \R, B^\ngr \vdash C^\ngr
                    }{
                        IH
                    }
                }
            $$
        \item $A=B \vee C$
            $$
                \infer{
                    \R \vdash \ngr\ngr(B^\ngr \vee C^\ngr)
                }{
                    \infer{
                        \R, \ngr(B^\ngr \vee C^\ngr) \vdash \R
                    }{
                    }
                }
            $$
        \item $A=\forall x B$
            $$
                \infer{
                    \R \vdash \forall x B^\ngr
                }{
                    \infer{
                        \R \vdash B^\ngr
                    }{
                        IH
                    }
                }
            $$
        \item $A=\exists x B$
            $$
                \infer{
                    \R \vdash \ngr\ngr(\exists x B^\ngr)
                }{
                    \infer{
                        \R, \ngr(\exists x B^\ngr) \vdash \R
                    }{
                    }
                }
            $$
    \end{itemize}
    So, $\R\vdash A^\ngr$
    
    \item We proof this property by induction on $A$ and we denote by $IH$ the induction hypothesis.
    \begin{itemize}
        \item $A=\bottom$
            $$
                \infer{
                    \ngr\ngr \bottom^\ngr \vdash \bottom^\ngr
                }{
                    \infer{
                        \ngr\ngr \R \vdash \R
                    }{
                        \infer{
                            \ngr\ngr \R \vdash \R\Rightarrow\R\Rightarrow\R
                        }{
                        }&
                        \infer{
                            \ngr\ngr \R \vdash \R\Rightarrow\R
                        }{
                            \infer{
                                \ngr\ngr \R, \R \vdash \R
                            }{
                            }
                        }
                    }
                }
            $$
        \item $A=\top$
            $$
                \infer{
                    \ngr\ngr \top^\ngr \vdash \top^\ngr
                }{
                    \infer{
                        \ngr\ngr \top^\ngr \vdash \top
                    }{
                    }
                }
            $$
        \item $A=(a \dot{=} b)$
            $$
            \infer{
                                (\Theta_1) : \ngr \ngr (a \dot{=} b)^\ngr, \ngr(a \dot{=} b), \ngr \ngr (a \dot{=} b) \vdash \R
                            }{
                                \infer{
                                    \ngr \ngr (a \dot{=} b)^\ngr, \ngr(a \dot{=} b), \ngr \ngr (a \dot{=} b) \vdash \ngr \ngr (a \dot{=} b)
                                }{
                                }&
                                \infer{
                                    \ngr(a \dot{=} b), \ngr \ngr (a \dot{=} b) \vdash \ngr (a \dot{=} b)
                                }{
                                }
                            }
            $$
            
            $$
                \infer{
                    \ngr \ngr (a \dot{=} b)^\ngr \vdash (a \dot{=} b)^\ngr
                }{
                    \infer{
                        \ngr \ngr (a \dot{=} b)^\ngr, \ngr(a \dot{=} b) \vdash \R
                    }{
                        \infer{
                            \ngr \ngr (a \dot{=} b)^\ngr, \ngr(a \dot{=} b) \vdash \ngr \ngr \ngr \ngr (a \dot{=} b)
                        }{
                        }&
                        \infer{
                            \ngr \ngr (a \dot{=} b)^\ngr, \ngr(a \dot{=} b) \vdash \ngr \ngr \ngr (a \dot{=} b)
                        }{
                            \Theta_1
                        }
                    }
                }
            $$
        \item $A=B\wedge C$
        
            $$
            \infer{
                            (\Theta_2) : \ngr\ngr(B \wedge C)^\ngr, \ngr B^\ngr \vdash \ngr(B \wedge C)^\ngr
                        }{
                            \infer{
                                \ngr\ngr(B \wedge C)^\ngr, \ngr B^\ngr, (B \wedge C)^\ngr \vdash \R
                            }{
                                \infer{
                                    \ngr\ngr(B \wedge C)^\ngr, \ngr B^\ngr, (B \wedge C)^\ngr \vdash \ngr B^\ngr
                                }{
                                }&
                                \infer{
                                    \ngr\ngr(B \wedge C)^\ngr, \ngr B^\ngr, (B \wedge C)^\ngr \vdash B^\ngr
                                }{
                                    \infer{
                                        \ngr\ngr(B \wedge C)^\ngr, \ngr B^\ngr, (B \wedge C)^\ngr \vdash B^\ngr \wedge C^\ngr
                                    }{
                                    }
                                }
                            }
                        }
            $$        
        
            $$
            \infer{
                    (\Theta_3) : \ngr\ngr(B \wedge C)^\ngr \vdash \ngr\ngr B^\ngr
                }{
                    \infer{
                        \ngr\ngr(B \wedge C)^\ngr, \ngr B^\ngr \vdash \R
                    }{
                        \infer{
                            \ngr\ngr(B \wedge C)^\ngr, \ngr B^\ngr \vdash \ngr\ngr(B \wedge C)^\ngr
                        }{
                        }&
                        \Theta_2
                    }
                }
            $$        
        
            $$
            \infer{
                (\Theta_4) : \ngr\ngr(B \wedge C)^\ngr \vdash B^\ngr
            }{
                \Theta_3
                &
                \infer{
                    \ngr\ngr(B \wedge C)^\ngr \vdash \ngr\ngr B^\ngr \Rightarrow  B^\ngr
                }{
                    \infer{
                        \ngr\ngr(B \wedge C)^\ngr, \ngr\ngr B^\ngr \vdash  B^\ngr
                    }{
                        IH_B
                    }
                }
            }
            $$
            
            $$
            \infer{
                            (\Theta_5) : \ngr\ngr(B \wedge C)^\ngr, \ngr C^\ngr \vdash \ngr(B \wedge C)^\ngr
                        }{
                            \infer{
                                \ngr\ngr(B \wedge C)^\ngr, \ngr C^\ngr, (B \wedge C)^\ngr \vdash \R
                            }{
                                \infer{
                                    \ngr\ngr(B \wedge C)^\ngr, \ngr C^\ngr, (B \wedge C)^\ngr \vdash \ngr C^\ngr
                                }{
                                }&
                                \infer{
                                    \ngr\ngr(B \wedge C)^\ngr, \ngr C^\ngr, (B \wedge C)^\ngr \vdash C^\ngr
                                }{
                                    \infer{
                                        \ngr\ngr(B \wedge C)^\ngr, \ngr C^\ngr, (B \wedge C)^\ngr \vdash B^\ngr \wedge C^\ngr
                                    }{
                                    }
                                }
                            }
                        }
            $$        
        
            $$
            \infer{
                    (\Theta_6) : \ngr\ngr(B \wedge C)^\ngr \vdash \ngr\ngr C^\ngr
                }{
                    \infer{
                        \ngr\ngr(B \wedge C)^\ngr, \ngr C^\ngr \vdash \R
                    }{
                        \infer{
                            \ngr\ngr(B \wedge C)^\ngr, \ngr C^\ngr \vdash \ngr\ngr(B \wedge C)^\ngr
                        }{
                        }&
                        \Theta_5
                    }
                }
            $$        
        
            $$
            \infer{
                (\Theta_7) : \ngr\ngr(B \wedge C)^\ngr \vdash C^\ngr
            }{
                \Theta_6
                &
                \infer{
                    \ngr\ngr(B \wedge C)^\ngr \vdash \ngr\ngr C^\ngr \Rightarrow  C^\ngr
                }{
                    \infer{
                        \ngr\ngr(B \wedge C)^\ngr, \ngr\ngr C^\ngr \vdash  C^\ngr
                    }{
                        IH_C
                    }
                }
            }
            $$
            
            $$
                \infer{
                    \ngr\ngr(B \wedge C)^\ngr \vdash (B \wedge C)^\ngr
                }{
                    \infer{
                        \ngr\ngr(B \wedge C)^\ngr \vdash B^\ngr \wedge C^\ngr
                    }{
                        \Theta_4
                        &
                        \Theta_7
                    }
                }
            $$
        \item $A=B \Rightarrow C$
            $$
                \infer{
                    \ngr\ngr (B \Rightarrow C)^\ngr \vdash (B \Rightarrow C)^\ngr
                }{
                    \infer{
                        \ngr\ngr (B \Rightarrow C)^\ngr \vdash B^\ngr \Rightarrow C^\ngr
                    }{
                        \infer{
                            \ngr\ngr (B \Rightarrow C)^\ngr, B^\ngr \vdash C^\ngr
                        }{
                            todo
                        }
                    }
                }
            $$
        \item $A=B \vee C$
            $$            
                    \infer{
                        (\Theta_8) : \ngr\ngr(B \vee C)^\ngr, \ngr(B^\ngr \vee C^\ngr), (B \vee C)^\ngr \vdash \ngr(B^\ngr \vee C^\ngr)
                    }{
                        \infer{
                            \ngr\ngr(B \vee C)^\ngr, \ngr(B^\ngr \vee C^\ngr), (B \vee C)^\ngr, B^\ngr \vee C^\ngr \vdash \R
                        }{
                            \infer{
                                \ngr(B^\ngr \vee C^\ngr) \vdash \ngr(B^\ngr \vee C^\ngr)
                            }{
                            }&
                            \infer{
                                B^\ngr \vee C^\ngr \vdash B^\ngr \vee C^\ngr
                            }{
                            }
                        }
                    }
            $$
            
            $$
            \infer{
                (\Theta_9) : \ngr\ngr(B \vee C)^\ngr, \ngr(B^\ngr \vee C^\ngr) \vdash \ngr(B \vee C)^\ngr
            }{
                \infer{
                    \ngr\ngr(B \vee C)^\ngr, \ngr(B^\ngr \vee C^\ngr), (B \vee C)^\ngr \vdash \R
                }{
                    \infer{
                        \ngr\ngr(B^\ngr \vee C^\ngr) \vdash \ngr\ngr(B^\ngr \vee C^\ngr)
                    }{
                    }&
                    \Theta_8
                }
            }
            $$
        
            $$
                \infer{
                    \ngr\ngr(B \vee C)^\ngr \vdash (B \vee C)^\ngr
                }{
                    \infer{
                        \ngr\ngr(B \vee C)^\ngr \vdash \ngr\ngr(B^\ngr \vee C^\ngr)
                    }{
                        \infer{
                            \ngr\ngr(B \vee C)^\ngr, \ngr(B^\ngr \vee C^\ngr) \vdash \R
                        }{
                            \infer{
                                \ngr\ngr(B \vee C)^\ngr, \ngr(B^\ngr \vee C^\ngr) \vdash \ngr\ngr(B \vee C)^\ngr
                            }{
                            }&
                            \Theta_9
                        }
                    }
                }
            $$
        \item $A=\forall x B$
            $$
                \infer{
                    \ngr\ngr \forall x B^\ngr \vdash \forall x B^\ngr
                }{
                    \infer{
                       \ngr\ngr \forall x B^\ngr \vdash B^\ngr
                    }{
                        IH
                    }
                }
            $$
        \item $A=\exists x B$
            $$\infer{
                            (\Theta_{10}) : \ngr\ngr(\exists x B)^\ngr, \ngr(\exists x B^\ngr) \vdash \ngr(\exists x B)^\ngr
                        }{
                            \infer{
                                \ngr\ngr(\exists x B)^\ngr, \ngr(\exists x B^\ngr), (\exists x B)^\ngr \vdash \R
                            }{
                                \infer{
                                    (\exists x B)^\ngr \vdash \ngr\ngr\exists x B^\ngr
                                }{
                                }&
                                \infer{
                                    \ngr(\exists x B^\ngr), (\exists x B)^\ngr \vdash \ngr\exists x B^\ngr
                                }{
                                    \infer{
                                        \ngr(\exists x B^\ngr), (\exists x B)^\ngr, \exists x B^\ngr \vdash \R
                                    }{
                                        \infer{
                                            \ngr(\exists x B^\ngr) \vdash \ngr(\exists x B^\ngr)
                                        }{
                                        }&
                                        \infer{
                                            \exists x B^\ngr \vdash \exists x B^\ngr
                                        }{
                                        }
                                    }
                                }
                            }
                        }
            $$
        
            $$
                \infer{
                    \ngr\ngr(\exists x B)^\ngr \vdash \ngr\ngr(\exists x B^\ngr)
                }{
                    \infer{
                        \ngr\ngr(\exists x B)^\ngr, \ngr(\exists x B^\ngr) \vdash \R
                    }{
                        \infer{
                            \ngr\ngr(\exists x B)^\ngr \vdash \ngr\ngr(\exists x B)^\ngr
                        }{
                        }&
                        \Theta_{10}
                    }
                }
            $$
    \end{itemize}
    So, $\ngr\ngr A^\ngr\vdash A^\ngr$
\end{enumerate}
\end{proof}

\begin{theoreme}
    If $\vdash A$ is derivable in classical predicate logic and if no free variable of \R\, occurs in the derivation, then $\vdash A^{\neg_\R}$ is derivable in intuitionisctic predicate logic.
\end{theoreme}
\begin{proof}
    Every rules except excluded middle are the same so we just keep them  (and we replace all expression $X$ to $X^\ngr$) in order to get a derivation of $\vdash A^{\neg_\R}$ from a derivation of $\vdash A$. It works because there is no free occurrences of $R$.

For the excluded middle rule we rewrite:
    $$
    \infer{
        \Gamma \vdash A
    }{
        \infer{
            \Gamma, \neg A \vdash \bottom
        }{
            \vdots
        }
    }
    $$
to:
    $$
    \infer{
        \Gamma \vdash A^\ngr 
    }{
        \infer{
            \Gamma \vdash (A \wedge \neg A)^\ngr
        }{
            \infer{
                \vdash (A \wedge \neg A)^\ngr
            }{
                \text{proposition 4 \textit{(i)}}
            }
        }
        &
        \infer{
            \Gamma, A^\ngr \vdash A^\ngr 
        }{}
        &
        \infer{
            \Gamma, \ngr A^\ngr \vdash A^\ngr 
        }{
            \infer{
                \Gamma, \ngr A^\ngr \vdash \bottom
            }{
                \vdots
            }
        }
    }
    $$
\end{proof}

\begin{theoreme}
    If $PA \vdash A$ and if no free variable of \R\, occurs in the derivation, then $HA \vdash A^{\neg_\R}$.
\end{theoreme}
\begin{proof}
    \begin{enumerate}
    \item Injectivity of $S$
    $$    
                                \infer{
                                    (\Xi_1) : \neg_\R\neg_\R S(x)\dot{=}S(y), \neg_\R x\dot{=}y, S(x)\dot{=}S(y)  \vdash x\dot{=}y
                                }{
                                    \infer{
                                        \neg_\R\neg_\R S(x)\dot{=}S(y), \neg_\R x\dot{=}y, S(x)\dot{=}S(y)  \vdash S(x)\dot{=}S(y) 
                                    }{
                                    }&
                                    \infer{
                                        \vdash S(x)\dot{=}S(y) \Rightarrow x\dot{=}y
                                    }{
                                    }
                                }
    $$
    
    
    $$
    \infer{
                            (\Xi_2) : \neg_\R\neg_\R S(x)\dot{=}S(y), \neg_\R x\dot{=}y \vdash \neg_\R S(x)\dot{=}S(y) 
                        }{
                            \infer{
                                \neg_\R\neg_\R S(x)\dot{=}S(y), \neg_\R x\dot{=}y, S(x)\dot{=}S(y)  \vdash\R 
                            }{
                                \infer{
                                    \neg_\R\neg_\R S(x)\dot{=}S(y), \neg_\R x\dot{=}y, S(x)\dot{=}S(y)  \vdash\neg_\R x\dot{=}y
                                }{
                                }&
                                \Xi_1
                            }
                        }
    $$    
    
    $$
    \infer{
        \vdash (\forall xy (S(x)\dot{=}S(y) \Rightarrow x\dot{=}y))^{\neg_\R}
    }{
        \infer{
            \vdash \forall xy (\neg_\R\neg_\R S(x)\dot{=}S(y) \Rightarrow \neg_\R\neg_\R x\dot{=}y)
        }{
            \infer{
                \vdash \neg_\R\neg_\R S(x)\dot{=}S(y) \Rightarrow \neg_\R\neg_\R x\dot{=}y
            }{
                \infer{
                    \neg_\R\neg_\R S(x)\dot{=}S(y) \vdash\neg_\R\neg_\R x\dot{=}y
                }{
                    \infer{
                        \neg_\R\neg_\R S(x)\dot{=}S(y), \neg_\R x\dot{=}y \vdash \R 
                    }{
                        \infer{
                            \neg_\R\neg_\R S(x)\dot{=}S(y) \vdash \neg_\R\neg_\R S(x)\dot{=}S(y) 
                        }{
                        }&
                        \Xi_2
                    }
                }
            }
        }
    }
    $$
    
    \item Non confusion
    
    $$
    \infer{
        \vdash (\forall x \neg S(x)\dot{=}0)^{\neg_\R}
    }{
        \infer{
            \vdash \forall x \neg_\R\neg_\R\neg_\R S(x)\dot{=}0
        }{
            \infer{
                \vdash \neg_\R\neg_\R\neg_\R S(x)\dot{=}0
            }{
                \infer{
                    \neg_\R\neg_\R S(x)\dot{=}0 \vdash \R
                }{
                   \infer{
                        \neg_\R\neg_\R S(x)\dot{=}0 \vdash \neg_\R\neg_\R S(x)\dot{=}0
                    }{
                    }&
                   \infer{
                        \neg_\R\neg_\R S(x)\dot{=}0 \vdash \neg_\R S(x)\dot{=}0
                    }{
                        \infer{
                           \neg_\R\neg_\R S(x)\dot{=}0, S(x)\dot{=}0\vdash \R 
                        }{
                            \infer{
                                \neg_\R\neg_\R S(x)\dot{=}0, S(x)\dot{=}0\vdash \bottom 
                            }{
                                \infer{
                                    \neg_\R\neg_\R S(x)\dot{=}0, S(x)\dot{=}0\vdash S(x)\dot{=}0\ 
                                }{   
                                }&
                                \infer{
                                    \vdash \neg S(x)\dot{=}0\ 
                                }{
                                    \infer{
                                        \vdash \neg \forall xS(x)\dot{=}0\ 
                                    }{
                                    }
                                }
                            }
                        }
                    }
                }
            }
        }
    }
    $$
    
    \item Induction Scheme
    
    $$
    \infer{
        (\Xi_3) : A[0/x]^{\neg_\R}, \forall y(A[y/x]^{\neg_\R} \Rightarrow A[S(y)/x]^{\neg_\R}) \vdash \forall y(A[y/x]^{\neg_\R} \Rightarrow A[S(y)/x]^{\neg_\R}) \Rightarrow A^{\neg_\R}
    }{
        \infer{
            \vdash A[0/x]^{\neg_\R} \Rightarrow \forall y(A[y/x]^{\neg_\R} \Rightarrow A[S(y)/x]^{\neg_\R}) \Rightarrow A^{\neg_\R}
        }{
        }&
        \infer{
            A[0/x]^{\neg_\R} \vdash A[0/x]^{\neg_\R}
        }{
        }
    }
    $$
    
    $$
    \infer{
        \vdash (A[0/x] \Rightarrow \forall y(A[y/x] \Rightarrow A[S(y)/x]) \Rightarrow \forall x A)^{\neg_\R}
    }{
        \infer{
            \vdash A[0/x]^{\neg_\R} \Rightarrow \forall y(A[y/x]^{\neg_\R} \Rightarrow A[S(y)/x]^{\neg_\R}) \Rightarrow \forall x A^{\neg_\R}
        }{
            \infer{
                A[0/x]^{\neg_\R} \vdash \forall y(A[y/x]^{\neg_\R} \Rightarrow A[S(y)/x]^{\neg_\R}) \Rightarrow \forall x A^{\neg_\R}
            }{
                \infer{
                    A[0/x]^{\neg_\R}, \forall y(A[y/x]^{\neg_\R} \Rightarrow A[S(y)/x]^{\neg_\R}) \vdash \forall x A^{\neg_\R}
                }{
                    \infer{
                        A[0/x]^{\neg_\R}, \forall y(A[y/x]^{\neg_\R} \Rightarrow A[S(y)/x]^{\neg_\R}) \vdash A^{\neg_\R}
                    }{
                        \Xi_3
                        &
                        \infer{
                            \forall y(A[y/x]^{\neg_\R} \Rightarrow A[S(y)/x]^{\neg_\R}) \vdash \forall y(A[y/x]^{\neg_\R} \Rightarrow A[S(y)/x]^{\neg_\R})
                        }{
                        }
                    }
                }
            }
        }
    }
    $$
\end{enumerate}
\end{proof}

\begin{theoreme}
    If $PA\vdash \forall x,\exists y:(a\dot{=}b)$ then $HA\vdash \forall x, \exists y: (a\dot{=}b)$.
\end{theoreme}
\begin{proof}
    Let us write $F \forall x,G$ where $G$ is a $\Sigma_1^0$ formula. By using $(\forall E)$, if $F$ if provable with PA, so $G$ too. As $G$ is $\sigma_1^0$, $G$ is provable with $HA$. By using $(\forall I)$, we deduce that $F$ is provable with HA too.
    %cf. http://www.lsv.ens-cachan.fr/~goubault/types005.html
\end{proof}

\end{document}
