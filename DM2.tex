\input{macros.tex}

\author{
    Marc \textsc{Chevalier}\\
    Thomas \textsc{Pellissier} \textsc{Tanon}}
\date{\today}
\title{\textsc{Friedman}'s translation}

\begin{document}

\maketitle

\section{\textsc{Friedman}'s Translation}

\begin{definition}
    Let $\R$ be a formula. The \textbf{parametrized negation} is 
    $$
        \ngr := A \Rightarrow \R
    $$
\end{definition}

We gather here some basic properties of the parametrized negation

\begin{proposition}
    In intuitionisctic logic,
    \begin{enumerate}[(i)]
        \item $B \Rightarrow \ngr A \vdash A \Rightarrow \ngr B$
        \item $A\vdash \ngr\ngr A$
        \item $A\Rightarrow B \vdash \ngr B \Rightarrow \ngr A$
        \item $A\Rightarrow B \vdash \ngr\ngr A \Rightarrow \ngr \ngr B$
        \item $\ngr \ngr \ngr A \vdash \ngr A$
    \end{enumerate}
\end{proposition}
\begin{proof}
        \begin{enumerate}[(i)]
        \item 
            $$
                \infer{
                    B \Rightarrow \neg_\R A \vdash A \Rightarrow \neg_\R B
                }{
                    \infer{
                        B \Rightarrow \neg_\R A, A \vdash \neg_\R B
                    }{
                        \infer{
                            B \Rightarrow \neg_\R A, A, B \vdash \R
                        }{
                            \infer{
                                B \Rightarrow \neg_\R A, A, B \vdash A
                            }{
                            }&
                            \infer{
                                B \Rightarrow \neg_\R A, A, B \vdash A\Rightarrow \R
                            }{
                                \infer{
                                    B \Rightarrow \neg_\R A, A, B \vdash B
                                }{
                                }&
                                \infer{
                                    B \Rightarrow \neg_\R A, A, B \vdash B \Rightarrow A \Rightarrow R
                                }{
                                }
                            }
                        }
                    }
                }
            $$
            
        \item 
            $$
                \infer{
                    A\vdash \neg_\R\neg_\R A
                }{
                    \infer{
                        A, \neg_\R A \vdash \R
                    }{
                        \infer{
                            A, \neg_\R A \vdash A\Rightarrow \R
                        }{
                        }&
                        \infer{
                            A, \neg_\R A \vdash A
                        }{
                        }
                    }
                }
            $$
        \item
            $$
                \infer{
                    A\Rightarrow B \vdash \neg_\R B \Rightarrow \neg_\R A
                }{
                    \infer{
                        A\Rightarrow B, \neg_\R B \vdash \neg_\R A
                    }{
                        \infer{
                            A\Rightarrow B, \neg_\R B,A \vdash \R
                        }{
                            \infer{
                                A\Rightarrow B, \neg_\R B,A \vdash B\Rightarrow \R
                            }{
                            }&
                            \infer{
                                A\Rightarrow B, \neg_\R B,A \vdash B
                            }{
                                \infer{
                                    A\Rightarrow B, \neg_\R B,A \vdash A\Rightarrow B
                                }{
                                }&
                                \infer{
                                    A\Rightarrow B, \neg_\R B,A \vdash A
                                }{
                                }
                            }
                        }
                    }
                }
            $$
        \item
            $$
                \infer{
                                        (\Pi_0) : A\Rightarrow B, \neg_\R\neg_\R A,\neg_\R B, A\vdash B
                                    }{
                                        \infer{
                                            A\Rightarrow B, \neg_\R\neg_\R A,\neg_\R B, A\vdash A\Rightarrow B
                                        }{
                                        }&
                                        \infer{
                                            A\Rightarrow B, \neg_\R\neg_\R A,\neg_\R B, A\vdash A
                                        }{
                                        }
                                    }
            $$

            \bigskip            
            
            $$
                \infer{
                                (\Pi_1) : A\Rightarrow B, \neg_\R\neg_\R A,\neg_\R B \vdash A\Rightarrow \R
                            }{
                                \infer{
                                    A\Rightarrow B, \neg_\R\neg_\R A,\neg_\R B, A\vdash \R
                                }{
                                    \infer{
                                        A\Rightarrow B, \neg_\R\neg_\R A,\neg_\R B, A\vdash B \Rightarrow \R
                                    }{
                                    }&
                                    \infer{
                                        A\Rightarrow B, \neg_\R\neg_\R A,\neg_\R B, A\vdash B
                                    }{
                                        \Pi_0
                                    }
                                }
                            }
            $$
            
            \bigskip
            
            $$
                \infer{
                    A\Rightarrow B \vdash \neg_\R\neg_\R A \Rightarrow \neg_\R \neg_\R B
                }{
                    \infer{
                        A\Rightarrow B, \neg_\R\neg_\R A \vdash \neg_\R \neg_\R B
                    }{
                        \infer{
                            A\Rightarrow B, \neg_\R\neg_\R A,\neg_\R B \vdash \R 
                        }{
                            \infer{
                                A\Rightarrow B, \neg_\R\neg_\R A,\neg_\R B \vdash A\Rightarrow \R \Rightarrow \R 
                            }{
                            }&
                            \infer{
                                A\Rightarrow B, \neg_\R\neg_\R A,\neg_\R B \vdash A\Rightarrow \R
                            }{
                                \Pi_1
                            }
                        }
                    }
                }
            $$
        \item
            $$
                            \infer{
                                (\Pi_2) : \neg_\R \neg_\R \neg_\R A, A, A\Rightarrow\R \vdash \R
                            }{
                                \infer{
                                    \neg_\R \neg_\R \neg_\R A, A, A\Rightarrow\R \vdash A\Rightarrow\R
                                }{
                                }&
                                \infer{
                                    \neg_\R \neg_\R \neg_\R A, A, A\Rightarrow\R \vdash A
                                }{
                                }
                            }
            $$        
        
            $$
                \infer{
                    \neg_\R \neg_\R \neg_\R A \vdash \neg_\R A
                }{
                    \infer{
                        \neg_\R \neg_\R \neg_\R A, A \vdash \R
                    }{
                        \infer{
                            \neg_\R \neg_\R \neg_\R A, A \vdash A\Rightarrow\R\Rightarrow\R\Rightarrow\R
                        }{
                        }&
                        \infer{
                            \neg_\R \neg_\R \neg_\R A, A \vdash A\Rightarrow\R\Rightarrow\R
                        }{
                            \infer{
                                \neg_\R \neg_\R \neg_\R A, A, A\Rightarrow\R \vdash \R
                            }{
                                \Pi_2
                            }
                        }
                    }
                }
            $$
    \end{enumerate}
\end{proof}

We now define the parametrized translation.
\begin{definition}
    Let $\R$ be a formula. The \textbf{parametrized negative translation} $A^\ngr$ is defined by induction on $A$ as follows:
    \begin{center}
    \begin{tabular}{cccc}
        $\bottom^\ngr := \R$ & $\top^\ngr := \top$ & \multicolumn{2}{c}{$(a\dot{=}b)^\ngr := \ngr\ngr(a\dot{=}b)$}\\
        \multicolumn{2}{c}{$(A\wedge B)^{\neg_R}:=A^\ngr \wedge B^\ngr$} & \multicolumn{2}{c}{$(A \Rightarrow B)^{\neg_R}:=A^\ngr \Rightarrow B^\ngr$}\\
        \multicolumn{4}{c}{$(A \vee B)^{\neg_R}:=\ngr\ngr(A^\ngr \vee B^\ngr)$}\\
        \multicolumn{2}{c}{$\forall x A)^\ngr := \forall x A^\ngr$} & \multicolumn{2}{c}{$\exists x A^\ngr := \ngr\ngr(\exists x A^\ngr)$}
    \end{tabular}
    \end{center}
\end{definition}

Note that $(\neg A)^\ngr = \ngr A^\ngr$. We gather here the basic properties of the parametrized translation.

\begin{proposition}
    In intuitionistic logic,
    \begin{enumerate}[(i)]
        \item $\vdash (A\vee \neg A)^\ngr$
        \item $\R \vdash A^\ngr$
        \item $\ngr\ngr A^\ngr \vdash A^\ngr$
    \end{enumerate}
\end{proposition}
\begin{proof}
    \begin{enumerate}[(i)]
    \item 
    $$
        \infer{
            \vdash (A\vee \neg A)^\ngr
        }{
            \infer{
                \ngr(A^\ngr\vee (\neg A)^\ngr) \vdash \R
            }{
                \infer{
                    \ngr(A^\ngr\vee (\neg A)^\ngr) \vdash (A^\ngr\vee (\neg A)^\ngr) \Rightarrow \R
                }{
                }&
                \infer{
                    (\H) : \ngr(A^\ngr\vee (\neg A)^\ngr) \vdash A^\ngr\vee \ngr A^\ngr
                }{
                }
            }
        }
    $$
    We proof $\H$ by induction on $A$.
    
    \begin{itemize}
        \item $A=\bottom$
            $$
                \infer{
                    \ngr(A^\ngr\vee (\neg A)^\ngr) \vdash \R \vee \ngr \R
                }{
                    \infer{
                        \ngr(A^\ngr\vee (\neg A)^\ngr) \vdash \ngr \R
                    }{
                        \infer{
                            \ngr(A^\ngr\vee (\neg A)^\ngr), \R \vdash \R
                        }{
                        }
                    }
                }
            $$
        \item $A=\top$
            $$
                \infer{
                    \ngr(A^\ngr\vee (\neg A)^\ngr) \vdash \top \vee \ngr\top
                }{
                    \infer{
                        \ngr(A^\ngr\vee (\neg A)^\ngr) \vdash \top
                    }{
                    }
                }
            $$
        \item $A=(a\dot{=}b)$
            $$
                \infer{
                    \ngr(A^\ngr\vee (\neg A)^\ngr) \vdash \ngr\ngr(a\dot{=}b) \vee \ngr\ngr\ngr (a\dot{=}b)
                }{
                    \infer{
                        \R, \ngr(a\dot{=}b) \vdash \R
                    }{
                    }
                }
            $$
        \item $A=B\wedge C$
            $$
                \infer{
                    \R \vdash B^\ngr \wedge C^\ngr
                }{
                    \infer{
                        \R \vdash B^\ngr
                    }{
                        IH
                    }&
                    \infer{
                        \R \vdash C^\ngr
                    }{
                        IH
                    }
                }
            $$
        \item $A=B \Rightarrow C$
            $$
                \infer{
                    \R \vdash B^\ngr \Rightarrow C^\ngr
                }{
                    \infer{
                        \R, B^\ngr \vdash C^\ngr
                    }{
                        IH
                    }
                }
            $$
        \item $A=B \vee C$
            $$
                \infer{
                    \R \vdash \ngr\ngr(B^\ngr \vee C^\ngr)
                }{
                    \infer{
                        \R, \ngr(B^\ngr \vee C^\ngr) \vdash \R
                    }{
                    }
                }
            $$
        \item $A=\forall x B$
            $$
                \infer{
                    \R \vdash \forall x B^\ngr
                }{
                    \infer{
                        \R \vdash B^\ngr
                    }{
                        IH
                    }
                }
            $$
        \item $A=\exists x B$
            $$
                \infer{
                    \R \vdash \ngr\ngr(\exists x B^\ngr)
                }{
                    \infer{
                        \R, \ngr(\exists x B^\ngr) \vdash \R
                    }{
                    }
                }
            $$
    \end{itemize}

    \item We proof this property by induction on $A$ and we denote by $IH$ the induction hypothesis.
    \begin{itemize}
        \item $A=\bottom$
            $$
                \infer{
                    \R \vdash \R
                }{
                }
            $$
        \item $A=\top$
            $$
                \infer{
                    \R \vdash \top
                }{
                }
            $$
        \item $A=(a\dot{=}b)$
            $$
                \infer{
                    \R \vdash (a\dot{=}b)^\ngr
                }{
                    \infer{
                        \R, \ngr(a\dot{=}b) \vdash \R
                    }{
                    }
                }
            $$
        \item $A=B\wedge C$
            $$
                \infer{
                    \R \vdash B^\ngr \wedge C^\ngr
                }{
                    \infer{
                        \R \vdash B^\ngr
                    }{
                        IH
                    }&
                    \infer{
                        \R \vdash C^\ngr
                    }{
                        IH
                    }
                }
            $$
        \item $A=B \Rightarrow C$
            $$
                \infer{
                    \R \vdash B^\ngr \Rightarrow C^\ngr
                }{
                    \infer{
                        \R, B^\ngr \vdash C^\ngr
                    }{
                        IH
                    }
                }
            $$
        \item $A=B \vee C$
            $$
                \infer{
                    \R \vdash \ngr\ngr(B^\ngr \vee C^\ngr)
                }{
                    \infer{
                        \R, \ngr(B^\ngr \vee C^\ngr) \vdash \R
                    }{
                    }
                }
            $$
        \item $A=\forall x B$
            $$
                \infer{
                    \R \vdash \forall x B^\ngr
                }{
                    \infer{
                        \R \vdash B^\ngr
                    }{
                        IH
                    }
                }
            $$
        \item $A=\exists x B$
            $$
                \infer{
                    \R \vdash \ngr\ngr(\exists x B^\ngr)
                }{
                    \infer{
                        \R, \ngr(\exists x B^\ngr) \vdash \R
                    }{
                    }
                }
            $$
    \end{itemize}
    So, $\R\vdash A$
\end{enumerate}
\end{proof}

\begin{theoreme}
    If $\vdash A$ is derivable in classical predicate logic and if no free variable of \R\, occurs in the derivation, then $\vdash A^{\neg_\R}$ is derivable in intuitionisctic predicate logic.
\end{theoreme}
\begin{proof}

\end{proof}

\begin{theoreme}
    If $PA \vdash A$ and if no free variable of \R\, occurs in the derivation, then $HA \vdash A^{\neg_\R}$.
\end{theoreme}
\begin{proof}
    \begin{enumerate}
    \item Injectivity of $S$
    $$    
                                \infer{
                                    (\Xi_1) : \ngr\ngr S(x)\dot{=}S(y), \ngr x\dot{=}y, S(x)\dot{=}S(y)  \vdash x\dot{=}y
                                }{
                                    \infer{
                                        \ngr\ngr S(x)\dot{=}S(y), \ngr x\dot{=}y, S(x)\dot{=}S(y)  \vdash S(x)\dot{=}S(y) 
                                    }{
                                    }&
                                    \infer{
                                        \vdash S(x)\dot{=}S(y) \Rightarrow x\dot{=}y
                                    }{
                                    }
                                }
    $$
    
    
    $$
    \infer{
                            (\Xi_2) : \ngr\ngr S(x)\dot{=}S(y), \ngr x\dot{=}y \vdash \ngr S(x)\dot{=}S(y) 
                        }{
                            \infer{
                                \ngr\ngr S(x)\dot{=}S(y), \ngr x\dot{=}y, S(x)\dot{=}S(y)  \vdash\R 
                            }{
                                \infer{
                                    \ngr\ngr S(x)\dot{=}S(y), \ngr x\dot{=}y, S(x)\dot{=}S(y)  \vdash\ngr x\dot{=}y
                                }{
                                }&
                                \Xi_1
                            }
                        }
    $$    
    
    $$
    \infer{
        \vdash (\forall xy (S(x)\dot{=}S(y) \Rightarrow x\dot{=}y))^\ngr
    }{
        \infer{
            \vdash \forall xy (\ngr\ngr S(x)\dot{=}S(y) \Rightarrow \ngr\ngr x\dot{=}y)
        }{
            \infer{
                \vdash \ngr\ngr S(x)\dot{=}S(y) \Rightarrow \ngr\ngr x\dot{=}y
            }{
                \infer{
                    \ngr\ngr S(x)\dot{=}S(y) \vdash\ngr\ngr x\dot{=}y
                }{
                    \infer{
                        \ngr\ngr S(x)\dot{=}S(y), \ngr x\dot{=}y \vdash \R 
                    }{
                        \infer{
                            \ngr\ngr S(x)\dot{=}S(y) \vdash \ngr\ngr S(x)\dot{=}S(y) 
                        }{
                        }&
                        \Xi_2
                    }
                }
            }
        }
    }
    $$
    
    \item Non confusion
    
    $$
    \infer{
        \vdash (\forall x \neg S(x)\dot{=}0)^\ngr
    }{
        \infer{
            \vdash \forall x \ngr\ngr\ngr S(x)\dot{=}0
        }{
            \infer{
                \vdash \ngr\ngr\ngr S(x)\dot{=}0
            }{
                \infer{
                    \ngr\ngr S(x)\dot{=}0 \vdash \R
                }{
                   \infer{
                        \ngr\ngr S(x)\dot{=}0 \vdash \ngr\ngr S(x)\dot{=}0
                    }{
                    }&
                   \infer{
                        \ngr\ngr S(x)\dot{=}0 \vdash \ngr S(x)\dot{=}0
                    }{
                        \infer{
                           \ngr\ngr S(x)\dot{=}0, S(x)\dot{=}0\vdash \R 
                        }{
                            \infer{
                                \ngr\ngr S(x)\dot{=}0, S(x)\dot{=}0\vdash \bottom 
                            }{
                                \infer{
                                    \ngr\ngr S(x)\dot{=}0, S(x)\dot{=}0\vdash S(x)\dot{=}0\ 
                                }{   
                                }&
                                \infer{
                                    \vdash \neg S(x)\dot{=}0\ 
                                }{
                                    \infer{
                                        \vdash \neg \forall xS(x)\dot{=}0\ 
                                    }{
                                    }
                                }
                            }
                        }
                    }
                }
            }
        }
    }
    $$
    
    \item Induction Scheme
    
    $$
    \infer{
        (\Xi_3) : A[0/x]^\ngr, \forall y(A[y/x]^\ngr \Rightarrow A[S(y)/x]^\ngr) \vdash \forall y(A[y/x]^\ngr \Rightarrow A[S(y)/x]^\ngr) \Rightarrow A^\ngr
    }{
        \infer{
            \vdash A[0/x]^\ngr \Rightarrow \forall y(A[y/x]^\ngr \Rightarrow A[S(y)/x]^\ngr) \Rightarrow A^\ngr
        }{
        }&
        \infer{
            A[0/x]^\ngr \vdash A[0/x]^\ngr
        }{
        }
    }
    $$
    
    $$
    \infer{
        \vdash (A[0/x] \Rightarrow \forall y(A[y/x] \Rightarrow A[S(y)/x]) \Rightarrow \forall x A)^\ngr
    }{
        \infer{
            \vdash A[0/x]^\ngr \Rightarrow \forall y(A[y/x]^\ngr \Rightarrow A[S(y)/x]^\ngr) \Rightarrow \forall x A^\ngr
        }{
            \infer{
                A[0/x]^\ngr \vdash \forall y(A[y/x]^\ngr \Rightarrow A[S(y)/x]^\ngr) \Rightarrow \forall x A^\ngr
            }{
                \infer{
                    A[0/x]^\ngr, \forall y(A[y/x]^\ngr \Rightarrow A[S(y)/x]^\ngr) \vdash \forall x A^\ngr
                }{
                    \infer{
                        A[0/x]^\ngr, \forall y(A[y/x]^\ngr \Rightarrow A[S(y)/x]^\ngr) \vdash A^\ngr
                    }{
                        \Xi_3
                        &
                        \infer{
                            \forall y(A[y/x]^\ngr \Rightarrow A[S(y)/x]^\ngr) \vdash \forall y(A[y/x]^\ngr \Rightarrow A[S(y)/x]^\ngr)
                        }{
                        }
                    }
                }
            }
        }
    }
    $$
\end{enumerate}
\end{proof}

\begin{theoreme}
    If $PA\vdash \forall x,\exists y:(a\dot{=}b)$ then $HA\vdash \forall x, \exists y: (a\dot{=}b)$.
\end{theoreme}
\begin{proof}
    Let us write $F \forall x,G$ where $G$ is a $\Sigma_1^0$ formula. By using $(\forall E)$, if $F$ if provable with PA, so $G$ too. As $G$ is $\sigma_1^0$, $G$ is provable with $HA$. By using $(\forall I)$, we deduce that $F$ is provable with HA too.
    %cf. http://www.lsv.ens-cachan.fr/~goubault/types005.html
\end{proof}

\end{document}
